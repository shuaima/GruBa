\section{Group based Behavior Modeling}
\label{sec:gm}

%clear points
%road map
%detail of each step, with motivation


\par Recall the central problem of \sys{}, where the \retg{} behaviors of users are modeled.
Specifically, such model is built by the Group Modeling module for each user group and thus named as group model.
To avoid ambiguity, we shall use the term of \textit{items} to denote the data for training the group model.
A given \textit{item} is either positive or negative.

\begin{definition}
\label{def:gm-it}
An item $E$ involves a microblog $b$ and a user $f$ such that $f\ \in R_{b.O}$, i.e., $f$ is a follower of  the owner of microblog $b$.
\begin{equation}
\label{eq:gm-it}
E \in
  \begin{cases}
    \text{positive items}       & \quad \text{if } f \text{ \retd{} } b\\
    \text{negative items}  		& \quad \text{if } f \text{ did not \ret{} } b
  \end{cases}
\end{equation}
\end{definition}

And the data of item $E$ consists of three parts.
%\begin{itemize}

	\stab(1) \stitle{User Info} contains a list of aforementioned metrics \{$G_u$, $P_u$,\#$R_u$, \#$E_u$, $R_{ee,u}$\}.
	
%	\stab(2) \stitle{Microblog Info} refers to metrics related to the microblog $b$. The number of times the $b$ be retweeted, be %commented, be liked and the length of the microblog contents are considered. What's more, we also considers the correlation between %microblog contents and recent events, where the latter is returned by Ring \cite{IEEEexample:ring}. The correlation metric $C_h$ is in %the form of a normalized vector with each dimension represents one event (similar as $P_{f,u}$ in formula \ref{eq:inte}). Each event %could be viewed as a topic $t$, over which the correlation of a microblog $b$ could be obtained by formula \ref{eq:sim-tw1}.

\stab(2) \stitle{Microblog Info} refers to metrics related to the microblog $b$.
%The number of times the $b$ has been retweeted, commented, liked and the length of the microblog contents are considered.
The number that $b$ has been retweeted, commented, liked and the length of $b.M$ (microblog message) are considered.
What is more, we consider the correlation between $b$ and recent event, where the latter is expressed as several core words returned by Ring \cite{IEEEexample:ring}.
Here we compute \textit{TF-IDF weight} $W_f$ of $b.M$, and get correlation metric $C_h$ of $b$ and event by formula \ref{eq:sim-tf1}.

%	\stab(3) \stitle{Interaction Info} includes seven correlation metrics. They are of \#$B_u$, $R_{oc,u}$, \#$W_{r,u}$, \#$W_{t,u}$, %microblog $b$ versus the user $u$'s \textit{Interest Feature} $P_f(u)$ (a.k.a. long-term/stable interest in this work), $b$ versus %$u$'s short-term interest $P_s(u)$ that is mined from $u$'s recent microblogs (e.g., within 30 days) in the same manner of $P_f(u)$, %and $b$'s timestamp versus the time distribution of $u$'s \retg{} behavior $P_{rt,u}$.
%\end{itemize}
	\stab(3) \stitle{Interaction Info} includes seven correlation metrics: \#$B_u$, $R_{oc,u}$, \#$W_{r,u}$, \#$W_{t,u}$, microblog $b$ versus the user $u$'s \textit{Interest Feature} $P_u$ (a.k.a. long-term/stable interest in this work), and $b$'s timestamp versus the time distribution of $u$'s \retg{} behavior $P_{rt,u}$.
%In addition, the correlation between microblog contents and $u$'s short-term interest mined from $u$'s recent microblogs (e.g., within 30 days) is also taken in consideration.
%In addition, the correlation between microblog contents and $u$'s short-term interest is also taken in consideration. The short-term %interest is expressed as
In addition, we consider $u$'s short-term interest, which is mined from $u$'s recent microblogs (e.g., within 30 days) and calculated by \textit{TF-IDF}, namely $W_s$.
The correlation between microblog $b$ and $u$'s short-term interest is computed by $W_f$ and $W_s$ (using formula \ref{eq:sim-tf1}).
%%

The modeling of retweeting behavior of groups is treated as a classification problem, and we utilize the random forest classifier to address it. Details for random forest \cite{IEEEexample:conf/icdar/Ho1995} are omitted here for space reason.
The advantage of this classification model lies in that it could integrate different features conveniently, and
%The advantage of using classification model is that we can integrate different combinations of the features into the model conveniently.
the obtained group behavior model could learn what a positive/negative item looks like over each metric mentioned above. \par
Here we use accuracy to evaluate our model. To define accuracy, we set four variables: $E_{tp}$,$E_{fp}$,$E_{tn}$ and $E_{fn}$. For a given item $E$, if $E$ is a positive item and our model also determines it a positive item, then we set $E_{tp}$ to 1, else we set $E_{tp}$ to 0. If $E$ is a negative item and our model determines it a positive item, then we set $E_{fp}$ to 1, else we set $E_{fp}$ to 0. If $E$ is a negative item and our model determines it a negative item, then we set $E_{tn}$ to 1, else we set $E_{tn}$ to 0. If $E$ is a positive item and our model determines it a negative item, then we set $E_{fn}$ to 1, else we set $E_{fn}$ to 0. So the accuracy can be defined as:

\begin{equation}
\label{eq:def-precision1}
accuracy = \frac{\sum_{\substack{E}} (E_{tp} + E_{tn} )}{ \sum_{\substack{E}} (E_{tp} + E_{tn} + E_{fp} + E_{fn}) }
\end{equation}


















