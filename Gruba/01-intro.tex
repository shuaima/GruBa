\section{Introduction}
\label{sec:intro}

Social media is overwhelming nowadays, with massive users on Facebook \refe{}, Twitter \refe{} and Weibo \refe{} while the number of users keeps increasing.
These users behave variously, knowledge of which is significant in recommendation system, activity prediction and Big Thing analysis.
Hence emerges the demand of developing systems and algorithms that could properly model user behaviors, which has attracted the attention from both academia and industry.

Central to user behavior modeling, is the need to choose the unit of model (i.e., how many users share one model), as well as the variety of features to be selected for differentiating these units.
Already, there exist work of building a single model for all the users \refe{}.
Apparently, such model bears the limitation of being coarse.
On the other hand, modeling each user is not practical, due to the tremendous number of users.

The key driver of our work is the realization that in social media applications, users could fall into groups and each group shares representative behaviors.
Particularly, we study the \retg{} behavior of users and our work can be generalized to other behaviors of like and comment as well. 
As one example, consider the film \textit{Brave Heart}, fans of which are probably addicted to highland, bagpipe and war films, and thus likely to \ret{} blogs of these topics.

The contributions of this work include:
\begin{itemize}
\item We present a system named \sys{} with the novel perspective to model user behaviors over groups instead of the mono model in literature.
\item We leverage user interests to facilitate the modeling of \retg{} behavior and look into interests with various dimensions, including long-term/recent interests and explicit/implicit interests.
\item We evaluate the performance of \sys{} using real-world datasets, showcasing its benefits against competitive state of the art approaches.
\end{itemize}


The rest of this paper is organized as follows.
Section \ref{sec:overv} first gives the problem formulation and subsequently overviews \sys{}'s components, principle of which are detailed in Sections \ref{sec:fe}, \ref{sec:uc} and \ref{sec:gm} separately.
Section \ref{sec:perf} provides the performance evaluation.
Related work is presented in Section \ref{sec:rela}.
Finally, Section \ref{sec:conclu} concludes the work.











