% jing @ 0927
% back to algorithm for interest features
% add motivation, intuition, etc

%clear points
%road map
%detail of each step, with motivation

\section{Feature Extraction}
\label{sec:fe}

With the underlying Sina Microblog Data, Feature Extraction is responsible for ``mining'' the user characteristics, resulting three features for each user.
These features are \textit{Basic Feature}, \textit{Behavior Feature} and \textit{Interest Feature}, which constitute the \textit{Feature Data} in \sys{}, i.e., \textit{Feature Data} = \{\textit{Basic Feature}, \textit{Behavior Feature}, \textit{Interest Feature}\}.

\subsection{Basic Feature}

\textit{Basic Feature} employs a vector $I$ to cover the basic info of user.
\begin{equation}
\label{eq:info}
	I = (\#R_u,\ \#E_u,\ R_{ee},\ \ U_t)
\end{equation}
where each variable is illustrated in Table \ref{tbl:fe-info}.
%Specifically, we use $\#(B_s | R(B)==1)$ and $\#(B_s | R(B)==0)$ to represent the number of \retd{} microblogs and microblogs that are originally created by the user.



\subsection{Behavior Feature}

Unlike \textit{Basic Feature}, \textit{Behavior Feature} shows several statistics regarding the user's \retg{} behavior.
Such statistics include:
\begin{itemize}
	\item a value for the number of owned microblogs $\#B_u$
	\item a ratio $R_{oc}$ that is defined as the number of \retd{} microblogs over that of originally created, i.e., $\frac{\#(B_u | flag ==1)}{\#(B_u | flag ==0)}$
	\item a value showing the average number of \retd{} microblogs per week: $\#W_r$
	\item a normalized vector regarding the time distribution of a user's \retg{} behavior: $P_t\ = (p'_0,\ p'_1,\ ...,\ p'_{11})$, where $p'_0$ is the probability that the \retg{} activity happens from 0am to 2am, $p'_1$ is the probability that the \retg{} activity happens from 2am to 4am, and so on.
	\item a normalized vector with respect to the gap distribution of a user's \retg{} behavior: $P_g\ = (p''_0,\ p''_1,\ ...,\ p''_{5})$, in which $p''_0$ is the probability that the gap between two \retd{} microblogs is within 1 min. Ditto for $p''_1$ (1 min to 1 hour), $p''_2$ (1 to 12 hours), $p''_3$ (12 to 24 hours), $p''_4$ (24 to 48 hours) and  $p''_5$ (more than 48 hours).
\end{itemize}

Hence, \textit{Behavior Feature} $H$ per user comes with:
\begin{equation}
\label{eq:beha}
	H = (\#W_r,\ P_t,\ P_g)
\end{equation}
where $\#W_r$, $P_t$ and $P_g$ are illustrated as above.

\subsection{Interest Feature}

Different from the straightforward notions of \textit{Basic Feature} and \textit{Behavior Feature}, \textit{Interest Feature} involves a process of labeling users by their interested topics.
In short, with a given lexicon(made by some professionals) consisting of several \textit{topics}, the interest feature of a user is a normalized vector, in which each dimension refers to the probability that the said user matches a \textit{topic}.

\begin{definition}
\label{def:lexi}
A lexicon $L$ consists of a set of topics $t$ such that each topic is associated with a set of cell words $c$. Each cell word depicts an aspect of the topic.
\end{definition}

\begin{definition}
\label{def:bw}
%used \label{def:blog} previously
With a given user $u$, each microblog $b \in B_u$ could be decomposed into a set of words $w$.
\end{definition}

\begin{definition}
\label{def:inte}
The interest feature of a given user is a normalized vector
\begin{equation}
\label{eq:inte}
P_f\ = (p_0,\ p_1,\ ...,\ p_{x-1})
\end{equation}
in which the said user matches $x$ \textit{topics} in lexicon and $p_i$ refers to the similarity of the user and each matched topic (interest). The definition of such similarity shall be detailed in each scenario (explicit/implicit interest analysis, towards words/topics, etc).
\end{definition}


Next, we shall present the ``mining'' process for interest features.
In short, \sys{} employs a well established lexicon to discover the explicit interests of users.
In case no proper explicit interests are found, TF-IDF and Twitter-LDA are leveraged to explore the implicit interests, during which word2vector participates to provide the similarity between two words.

In \sys{}, a word, either in the form of $c$ or $w$, acts as the minimum unit for analysis.
Hence, the similarity of a word pair ($w$, $c$), i.e., $sim(w, c)$, could be generalized to the similarity of a microblog against one topic $sim(b, t)$, and finally to a user versus each topic in lexicon $sim(u, t)$; topics with similarity satisfying certain thresholds are allocated to the user $u$ and constitute the interests of $u$.

For instance, the following steps depict the ``mining'' process of the interest feature $P_f$ for a user $u$.

\stitle{Step 1:} Each microblog of $u$ is decomposed into a word set. %, i.e., $b = \{w\}$ where $b \in B_s(u)$.

\stitle{Step 2:} Explicit interests are explored. Specifically, every word $w$ is sent to match each cell word $c$ of lexicon topics.
If $w$ and $c$ are identical, $sim(w, c) = 1$.
Otherwise, $sim(w, c) = 0$.
As to the similarity of $b$ against a lexicon topic $t$, it is:
\begin{equation}
\label{eq:bt}
sim(b, t) = \sum_{\substack{i, j}} sim(w_i, c_j)
\end{equation}
where $sim(w_i, c_j)$ refers to the similarity of a word pair.

If $sim(b, t)$ satisfies a certain threshold (3 by default) , topic $t$ is labeled to microblog $b$;
the user $u$ is then discovered having an explicit interest (topic) $t$.
Thus, by looking into the similarity of $b$ against all topics in lexicon, the explicit interests of $u$ is returned, in the form of interest feature (see Definition \ref{def:inte}).

If none of $sim(b, t)$ could meet the threshold, i.e., explicit interest discovery over user $u$ fails, go to \textbf{Step 3} and \textbf{Step 4} in parallel, so as to ``mine'' the implicit interests of $u$.

\stitle{Step 3:} A metric \textit{TF-IDF weight} $W_f$ is computed, i.e., employing TF-IDF (term-frequency and inverse document-frequency) to calculate the weight distribution of words in microblog $b$:
\begin{equation}
\label{eq:tf}
W_f = \{(w_i, p_i)\}
\end{equation}
where $w_i$ refers to a single word, of which the weight is $p_i$, with $\sum_{\substack{i}} p_i = 1$.

To compute such weight $p_i$ for word $w_i$, a metric $p''_i$ is first calculated as:
\begin{equation}
\label{eq:tf-w}
p''_i = \frac{|b_i|}{|b|} * log(\frac{|D_i|}{|D|})
\end{equation}
in which we use the operator $|\ |$ to measure the cardinality, such that $|b_i|$ is the occurrences of word $w_i$ in microblog $b$ and $|b|$ the total occurrences of all words in $b$.
Ditto for $|D_i|$ and $|D|$, except that the scope is the overall dataset, rather than a single microblog $b$.

Hence, each word $w_i$ shall get an initial weight of $p''_i$, upon which the normalization is performed and $p_i$ is obtained, resulting the \textit{TF-IDF weight} (see Definition \ref{eq:tf}).
Go to \textbf{Step 5}.

\stitle{Step 4:} Similarly, another metric \textit{Twitter-LDA weight} $W_w$ is obtained, i.e., using Twitter-LDA  %\ref{IEEEexample:zhao2011comparing}
\cite{IEEEexample:zhao2011comparing}
to result the word weight distribution of microblog $b$.
Unlike TF-IDF, Twitter-LDA first trains the overall microblogs, allocating each microblog with a \textit{tag}.
%
The structure of \textit{tag} is as follows:
\begin{equation}
\label{eq:tw-tag}
W_t = \{(w'_i, p'_i)\}
\end{equation}
where $w'_i$ refers to a word in \textit{tag} $W_t$, and $p'_i$ is the probability that $w'_i$ appears in microblogs with the said \textit{tag}, with $\sum_{\substack{i}} p'_i = 1$ ($|W_t|\ =\ 30$ in this work by default).
%
Subsequently, $W_t$ are leveraged to conclude $W_w$, i.e., $W_w\ =\ W_t$ , which shares the format with that of $W_f$.
Go to \textbf{Step 6}.

\stitle{Step 5:} TF-IDF based similarity is calculated.
For example, the similarity (in the form of a value) of $W_f$ over a single topic $t$ in lexicon, written as $sim(W_f, t)$, is defined as:
\begin{equation}
\label{eq:sim-tf1}
sim(W_f, t) = \sum_{\substack{i}} p_i*sim(w_i, t)
\end{equation}
where $W_f = \{(w_i, p_i)\}$, $t = \{c_j\}$, and $sim(w_i, t)$ is the averaged word similarity $sim(w_i, c_j)$ returned by word2vector \cite{IEEEexample:mikolov2013distributed}.
Go to \textbf{Step 7}.

\begin{comment}
:
\begin{equation}
\label{eq:sim-tf2}
sim(w_i, t) = \sum_{\substack{j}} sim(w_i, c_j)
\end{equation}
\end{comment}



\stitle{Step 6:} Accordingly, Twitter-LDA based similarity is available.
Again, a single topic $t$ in lexicon is used for yardstick and the similarity of $W_w$ over $t$, written as $sim(W_w, t)$, is defined as:
\begin{equation}
\label{eq:sim-tw1}
sim(W_w, t) = \sum_{\substack{i}} p'_i*sim(w'_i, t)
\end{equation}
%where $W_w = \{(w'_i, p'_i)\}$, $t = \{c_j\}$,
where $W_w$ is a set of $(w'_i, p'_i)$, $t$ covers each $c_j$,
and $sim(w'_i, t)$ is the averaged word similarity $sim(w'_i, c_j)$ returned by word2vector.
Go to \textbf{Step 7}.

\begin{comment}
:
\begin{equation}
\label{eq:sim-tw2}
sim(w'_i, t) = \sum_{\substack{j}} sim(w'_i, c_j)
\end{equation}
\end{comment}



\stitle{Step 7:} Hence, the similarity of a microblog $b$ against a lexicon topic $t$ is given by:
\begin{equation}
\label{eq:simbt}
sim(b, t) = \alpha * sim(W_f, t) + (1 - \alpha) * sim(W_t, t)
\end{equation}
where the $\alpha$ is a parameter by which \sys{} could set flexible priorities between TF-IDF and Twitter-LDA.
Go to \textbf{Step 8}.

\stitle{Step 8:} Repeat the above steps (Step 1 to Step 7) for the microblog $b$ over every topic in lexicon, i.e., $\forall t_k \in L$ results one similarity value of $sim(b, t_k)$.
Such computation further extends to all the microblogs owned by user $u$, such that:
$\forall b_m \in B_s(u)$, $\forall t_k \in L$, there exists a similarity of $sim(b_m, t_k)$.
Hence, the overall similarity of user $u$ over lexicon topics $\{t\}$ (i.e., $L$), written as $S(u,\ L)$, could be denoted by a vector:
\begin{equation}
\label{eq:simul}
S(u, L) = (s_0,\ s_1,\ ...,\ s_{n-1})
\end{equation}
where $n$ refers to the cardinality of $L$ (i.e., number of topics in $L$) and $s_k$ is the overall similarity of user $u$ over topic $t_k$, which is given by:
\begin{equation}
\label{eq:simul-2}
s_k = \sum_{\substack{m}} sim(b_m, t_k)
\end{equation}

Among the $n$ dimensions of $S(u,\ L)$, those with top $x$ (3 in \sys{}) similarity values are selected to label the implicit interests of user $u$, which results an $x$ dimensional vector $P_f$ as described in Definition \ref{def:inte}.
Similarly, interest features of all users are returned.

As a result, the \textit{Feature Data} for every user $u$, written as $F(u)$, is given by:
\begin{equation}
\label{eq:fu}
	F(u) = (I,\ H,\ P_f)
\end{equation}
where $I$, $H$ and $P_f$ refer to \textit{Basic Feature}, \textit{Behavior Feature} and \textit{Interest Feature} separately (see formulas \ref{eq:info}, \ref{eq:beha} and \ref{def:inte}).
And it could be written as a vector:
\begin{equation}
\label{eq:fu-flat}
	F(u) = (\#R_s, \#E_s, R_{ee}, \#B_s, R_{oc}, U_t, \#W_r, P_t, P_g, P_f)
\end{equation}
where each dimension refers to a data item of \sys{}.

