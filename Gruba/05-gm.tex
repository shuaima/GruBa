\section{Group based Behavior Modelling}
\label{sec:gm}

%clear points
%road map
%detail of each step, with motivation

Recall the central problem of \sys{}, where the \retg{} behaviors of users are modeled.
Specifically, such model is built by Group Modeler for each user group and thus named as group model.
To avoid ambiguity, we shall use the term of \textit{items} to denote the data for training the group model.
A given \textit{item} is either positive or negative.

\begin{definition}
\label{def:gm-it}
An item $E$ involves a blog $b$ and a user $f$ such that $f\ \in R_s(O(b))$, i.e., $f$ is a follower of  the owner of blog $b$.
\begin{equation}
\label{eq:gm-it}
E \in
  \begin{cases}
    \text{positive items}       & \quad \text{if } f \text{ \retd{} } b\\
    \text{negative items}  		& \quad \text{if } f \text{ did not \ret{} } b
  \end{cases}
\end{equation}
\end{definition}

And the data of item $E$ could be further divided into three parts.
%\begin{itemize}

	\stab(1) \stitle{User Info} contains a list of aforementioned metrics \{\#$R_s$, \#$E_s$, $R_{ee}$, \#$B_s$, \#$W_r$\}.
	
	\stab(2) \stitle{Blog Info} considers the correlation between blog contents and recent events, where the latter is returned by Ring \cite{IEEEexample:ring}. The correlation metric $C_h$ is in the form of a normalized vector with each dimension represents one event (similar as $P_f$ in formula \ref{eq:inte}). Each event could be viewed as a topic $t$, over which the correlation of a blog $b$ could be obtained by formula \ref{eq:simbt}.
	
	\stab(3) \stitle{Interaction Info} includes three correlation metrics. They are of blog $b$ versus the user $u$'s \textit{Interest Feature} $P_f(u)$ (a.k.a. long-term/stable interest in this work), $b$ versus $u$'s short-term interest $P_s(u)$ that is mined from $u$'s recent blogs (e.g., within 30 days) in the same manner of $P_f(u)$, and $b$'s timestamp versus the time distribution of $u$'s \retg{} behavior $P_t$.
%\end{itemize}
As a result, the obtained group model could learn what does a positive/negative item look like over each metric mentioned above.









