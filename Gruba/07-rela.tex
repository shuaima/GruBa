\section{\textcolor{red}{Related Work}}
\label{sec:rela}

%zhu begin
The rapid development of social network is accompanied by the generation of tremendous user-generated contents. Message forwarding (e.g., reposting on weibo.com) is one of the most popular functions in many existing social networks. Behavior modeling of social media users has received great attention in recent years, and has become a research hot spot of academies and industries. In this section, we will introduce the related work of user modeling in social network from the aspects of user features analysis, user groups mining and user behaviors modeling.\par

Social media users's features we mentioned here include users's basic information, behavior features and interests features. Users's basic information include the users's gender, age, region, occupation and other personal information. There are many researches for analyzing users's basic information, such as users's racial information analysis \cite{IEEEexample:conf/icwsm/PennacchiottiP11}, users's gender inference \cite{IEEEexample:conf/emnlp/CiotSR13}, users's actual age inference \cite{IEEEexample:conf/icde/ParkHHL09}, user political tendency analysis (e.g. \cite{IEEEexample:conf/icwsm/PennacchiottiP11}, \cite{IEEEexample:conf/acl/VolkovaCD14}), user policy orientation analysis\cite{IEEEexample:kosinski2013private}, user's geo-location and occupation mining\cite{IEEEexample:journals/tmm/FangSXH15,IEEEexample:conf/icde/FanCTWC16}, etc. Most of these methods analyze the unknown attributes by using classification or regression model.\par

Social medias users's behaviors mainly refers to the reposting, posting and commenting behaviors. Users's behaviors also have certain characteristics and regularity. As mentioned in \cite{IEEEexample:jiang2013understanding}, it found that the behavior of a user exhibits the power-law distribution at the time interval, and the power-law distribution is related to users's schedule. Guo Z et al.\cite{IEEEexample:conf/music/GuoLTL12} analyzed the behavior of microblog users, the difference between the activity of users in different periods, and obtained the distribution of individual behavior on time.\par
Users's behaviors are largely driven by user interests, so it's of vital importance to model users's interests. Zhiyuan et al.\cite{IEEEexample:journals/fcsc/LiuCS12} modeled users's interests through mining keywords. They extracted keywords from the users's microblogs by the combination of words frequency and translation model. Xu Z et al.\cite{IEEEexample:conf/webi/XuLXY11} proposed a method which extended user topic model to analyze users's interests. Michelson M et al.\cite{IEEEexample:conf/and/MichelsonM10} analyzed interests  based on a knowledge base. They used a knowledge base to identify and classify the entities in twitters of one user, then generate the user's interests category subtree to express his interests. Lim K H et al.\cite{IEEEexample:conf/wikis/LimD13} analyzed the celebrities a user mentioned, then they got the preference degree of the user in different interests categories. Bhattacharya P et al.\cite{IEEEexample:conf/recsys/BhattacharyaZGGG14} proposed a method to extract a user's interests by analyzing the experts he followed. By digging a list of certain topics of the custom experts the user follows, they got the user's interests profiling. Wei Feng et al.\cite{IEEEexample:conf/icde/FengW14} studied the methods mapping tweets to hashtags to get users's preferences for hashtags.\par

Considering the fact that the amount of users in Social Medias is so huge, modeling for each user is not a good idea. In addition, modeling for a single user may make the model too particular. There are also some studies on the user groups analysis. Some researches studied how to classify users under a specific situation. For instance, Marco Pennacchiotti et al.\cite{IEEEexample:conf/icwsm/PennacchiottiP11} classified users by race, political tendencies and so on. In addition, research for user community mining is a hot spot. Jaewon Yang et al.\cite{IEEEexample:conf/wsdm/YangL13} proposed a method based on non-negative matrix decomposition to mine user community. Yiye Ruan et al.\cite{IEEEexample:conf/www/RuanFP13} considered user's friends and user's text content into their method when measuring the similarity between users for clustering. He et al. \cite{IEEEexample:he2014overlapping}
only considered the relationship between friends, and used the edge aggregation coefficient as a measure of clustering. Hiroaki Shiokawa et al.\cite{IEEEexample:conf/aaai/ShiokawaFO13} used the modular degree as a clustering standard and proposed an incremental algorithm to mine user community.\par

As for behavior prediction, Jing Zhang et al.\cite{IEEEexample:journals/tkdd/ZhangTLLX15} proposed a method of analyzing users's reposting behavior from the perspective of influence. Some researchers (e.g. \cite{IEEEexample:conf/wsdm/FengW13}, \cite{IEEEexample:conf/ijcai/ZhangLTCL13}) find that the user's reposting behavior is largely influenced by the relationship between friends, so they consider user's friend structure information into their models. Bo Jiang et al. \cite{IEEEexample:conf/cikm/JiangLSW15} proposed a method that combined matrix decomposition and microblogs clustering to analyze the user's reposting behavior. Zhang et al.\cite{IEEEexample:zhang2015retweet} proposed non-parametric statistical models to combine structural, textual, and temporal information together to predict reposting behavior. Considering the situation that users's not reposting does not mean that users are not interested in it, the users may just did not see it, \cite{IEEEexample:conf/sigir/JiangLSLLMW16} applied a collaborative filtering method to the reposting behavior analysis.
