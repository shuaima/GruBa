\section{Related Work}
\label{sec:rela}

%In this section, we will introduce the related work of user modeling in social network from the aspects of user features analysis, user groups mining and user behaviors modeling. Social media users's features we mentioned here include users's basic information, behavior features and interests features.\par

%Users's basic information we mentioned here include the users's gender, age, region, occupation and other personal information. There are many researches for analyzing users's basic information, such as users's racial information analysis \cite{IEEEexample:conf/icwsm/PennacchiottiP11}, users's gender inference \cite{IEEEexample:conf/emnlp/CiotSR13}, users's actual age inference \cite{IEEEexample:conf/icde/ParkHHL09}, user policy orientation analysis (e.g. \cite{IEEEexample:conf/icwsm/PennacchiottiP11,IEEEexample:conf/acl/VolkovaCD14,IEEEexample:kosinski2013private}), user's geo-location and occupation mining \cite{IEEEexample:journals/tmm/FangSXH15,IEEEexample:conf/icde/FanCTWC16}, etc. Here, we did not do basic information mining for users. We used the user's basic information as important features for user clustering and behavior modeling. \par

%Social medias users's behaviors mainly refers to the tweeting and retweeting behaviors. Users's behaviors also have certain characteristics and regularity, which have been proved by a lot of existing work. Jiang et al. proposed a behavior dynamics model \cite{IEEEexample:jiang2013understanding}, theoretical analysis shows this model can properly explain various heavy-tailed inter-event time distributions, including a regular power law and some non-power-law deviations. Guo Z et al.\cite{IEEEexample:conf/music/GuoLTL12} analyzed the behavior of microblog users, the difference between the activity of users in different periods, and obtained the distribution of individual behavior on time. Pravallika Devineni et al. analyzed the wall activities of users focusing on identifying common patterns and proposed PowerWall distribution \cite{IEEEexample:journals/snam/DevineniKFF17}. What's more, it helped them spot surprising behaviors and anomalies. Considering the regularity of user behavior on time, we extracted characteristics of the user behavior as important features for user clustering and behavior modeling.\par

%As for users's interest, many methods has been proposed to do interests extraction. Zhiyuan et al.\cite{IEEEexample:journals/fcsc/LiuCS12} modeled users's interests through mining keywords. They extracted keywords from the users's microblogs by the combination of words frequency and translation model. Xu Z et al.\cite{IEEEexample:conf/webi/XuLXY11} proposed a method which extended user topic model to analyze users's interests. Michelson M et al.\cite{IEEEexample:conf/and/MichelsonM10} analyzed interests  based on a knowledge base. They used a knowledge base to identify and classify the entities in twitters of one user, then generate the user's interests category subtree to express his interests. Lim K H et al.\cite{IEEEexample:conf/wikis/LimD13} analyzed the celebrities a user mentioned, then they got the preference degree of the user in different interests categories. Bhattacharya P et al.\cite{IEEEexample:conf/recsys/BhattacharyaZGGG14} proposed a method to extract a user's interests by analyzing the experts he followed. By digging a list of certain topics of the custom experts the user follows, they got the user's interests profiling. Wei Feng et al.\cite{IEEEexample:conf/icde/FengW14} studied the methods mapping tweets to hashtags to get users's preferences for hashtags.  The existing methods for interest extraction have some shortcomings. The granularity of the result obtained by the methods based on keyword extraction is too small, and the result obtained by the user topic model is implicit and difficult to display. What's more, for the lack of a complete knowledge base, it's not easy to do interests extraction by using knowledge base. To solve the problems, we employed a cell lexicon to express user's interest. What's more, we combined Twitter-LDA \cite{IEEEexample:zhao2011comparing} and TF-IDF to extract the distribution of users's interest.\par

%There have also many studies on the user groups analysis. Some researches studied how to classify users under a specific situation. For instance, Marco Pennacchiotti et al.\cite{IEEEexample:conf/icwsm/PennacchiottiP11} classified users by race, political tendencies and so on.  \cite{IEEEexample:journals/tkdd/ZhangCFLYZY17} analyzed the evolution of social groups, such as Facebook groups and Wechat groups, and proposed a new model for group evolution, which can provide insights about different evolution patterns of social groups. Xin Wang et al. \cite{IEEEexample:conf/aaai/WangDNGEB16} implemented a time-varying factorization to measure the user-group affinity for recommending groups to users. In addition, research for users community mining is a hot spot. Jaewon Yang et al.\cite{IEEEexample:conf/wsdm/YangL13} proposed a method based on non-negative matrix decomposition to mine user community. Yiye Ruan et al.\cite{IEEEexample:conf/www/RuanFP13} considered user's friends and user's text content into their method when measuring the similarity between users for clustering. He et al. \cite{IEEEexample:he2014overlapping}
%only considered the relationship between friends, and used the edge aggregation coefficient as a measure of clustering. Hiroaki Shiokawa et al.\cite{IEEEexample:conf/aaai/ShiokawaFO13} used the modular degree as a clustering standard and proposed an incremental algorithm to mine user community. As we can see, current analysis of user groups are diverse. However, users in social networks have multiple features, and many existing methods do not take advantage of these features to group users. Here we took user's basic, behavioral and interest information into account when performing user clustering, and we got distinguish groups with different statistical characteristics.\par

%As for behavior modeling, Jing Zhang et al.\cite{IEEEexample:journals/tkdd/ZhangTLLX15} proposed a method of analyzing users's retweeting behavior from the perspective of influence. Some researchers (e.g. \cite{IEEEexample:conf/wsdm/FengW13}, \cite{IEEEexample:conf/ijcai/ZhangLTCL13}) find that the user's reposting behavior is largely influenced by the relationship between friends, so they consider user's friend structure information into their models. Bo Jiang et al. \cite{IEEEexample:conf/cikm/JiangLSW15} proposed a method that combined matrix decomposition and microblogs clustering to analyze the user's reposting behavior. Zhang et al.\cite{IEEEexample:zhang2015retweet} proposed non-parametric statistical models to combine structural, textual, and temporal information together to predict reposting behavior. Considering the situation that users's not reposting does not mean that users are not interested in it, the users may just did not see it, \cite{IEEEexample:conf/sigir/JiangLSLLMW16} applied a collaborative filtering method to the reposting behavior analysis. Maria Giatsoglou et al. explore the identification of fraudulent and genuine retweet threads and developed a realistic generator that mimics the behaviors of both honest and fraudulent users \cite{IEEEexample:conf/pakdd/GiatsoglouCSFV15}. Suhas Ranganath et al. \cite{IEEEexample:journals/corr/RanganathMHTL15} were inspired by sociological theories of protest participation and proposed a framework to predict from the user¡¯s past status messages and interactions whether the next post of the user will be a declaration of protest.  They evaluated the framework using data from Twitter on protests during the recent Nigerian elections and demonstrated that it could effectively work. Considering the fact that the amount of users in Social Medias is so huge, modeling for each user is not a good idea. However, modeling for a single user may make the model too particular. In addition, modeling for the whole users makes our model inaccuracy.
%So we divided users into several groups by users clustering. Then we get different behavior model for each group users respectively. \par


In this section, we shall review related work in literature mainly from the aspects of analyzing features, mining groups and modeling behavior within the realm of social network modeling.
As aforementioned, \sys{} leverages the user features of basic info, behavior and interest.

With respect to feature analysis, there have been existed works of mining users' info, such as race \cite{IEEEexample:conf/icwsm/PennacchiottiP11}, gender \cite{IEEEexample:conf/emnlp/CiotSR13}, age \cite{IEEEexample:conf/icde/ParkHHL09}, political preference \cite{IEEEexample:conf/icwsm/PennacchiottiP11,IEEEexample:conf/acl/VolkovaCD14,IEEEexample:kosinski2013private} and occupation \cite{IEEEexample:journals/tmm/FangSXH15,IEEEexample:conf/icde/FanCTWC16}.
Our work, however, does not focus on the mining process per se; we use the mined info as the input for user clustering and group modeling.

Studies of behavior analysis put emphasis on exploring the characteristics.
For example, \cite{IEEEexample:jiang2013understanding} proposed a model that can properly explain various time distributions of user behaviors by theoretical analysis;
\cite{IEEEexample:conf/music/GuoLTL12} studied the user activity distribution of one day/week;
\cite{IEEEexample:journals/snam/DevineniKFF17} provided the PowerWall distribution of Facebook users, identifying a number of surprising behaviors and anomalies.
Considering the behavior characteristics, \sys{} makes use of them to feed the modeling process.

There have been established work of extracting user interests.
\cite{IEEEexample:journals/fcsc/LiuCS12} mined the user interests by exploring keywords of microblogs with the aid of word frequency and machine translation.
\cite{IEEEexample:conf/webi/XuLXY11} proposed a method of extending the topic model to obtain use interests.
Also, \cite{IEEEexample:conf/and/MichelsonM10} used a knowledge base and \cite{IEEEexample:conf/icde/FengW14} provided a solution of using hashtag for interest analysis.
\cite{IEEEexample:conf/wikis/LimD13} summarized user interest by exploring the mentioned celebrities;
Similarly, \cite{IEEEexample:conf/recsys/BhattacharyaZGGG14} leveraged the followed experts to result interest characteristics.
Unlike the existed solutions, \sys{} employs a cell lexicon to properly express user interest in which Twitter-LDA \cite{IEEEexample:zhao2011comparing} and TF-IDF are employed.

Approaches of grouping users in social network could fall into a variety of categories.
\cite{IEEEexample:conf/icwsm/PennacchiottiP11} grouped users by the info of race, political view and etc.
\cite{IEEEexample:journals/tkdd/ZhangCFLYZY17} studied the social groups on Facebook and Wechat, resulting various patterns of group evolution.
\cite{IEEEexample:conf/aaai/WangDNGEB16} proposed a time-varying factor to measure the affinity between users and groups such that proper group proposals are recommended.
More recent studies also look into mining user communities.
\cite{IEEEexample:conf/wsdm/YangL13} employed matrix decomposition to mine user community;
\cite{IEEEexample:conf/www/RuanFP13} and \cite{IEEEexample:he2014overlapping} considered followees info for user clustering;
\cite{IEEEexample:conf/aaai/ShiokawaFO13} proposed an incremental algorithm to mine user community using modular degree as the clustering yardstick.
Providing the diversity of user features, \sys{} employs feature of info, behavior and interest into user clustering.

The main problem of current approaches for behavior modeling lies in that the model is for either the overall users or a single user.
\cite{IEEEexample:journals/tkdd/ZhangTLLX15, IEEEexample:conf/wsdm/FengW13, IEEEexample:conf/ijcai/ZhangLTCL13} discovered that users' \retg{} behavior is largely influenced by their followees, whereas \cite{IEEEexample:conf/cikm/JiangLSW15} employed matrix decomposition, \cite{IEEEexample:conf/sigir/JiangLSLLMW16} used collaborative filtering methods and \cite{IEEEexample:zhang2015retweet} leveraged statistical models for \retg{} analysis.
\cite{IEEEexample:conf/pakdd/GiatsoglouCSFV15} and \cite{IEEEexample:journals/corr/RanganathMHTL15} focused on identifying whether the \retg{} is fraudulent or of protest.
Whereas our work \sys{} builds the \retg{} model for each user group, instead of the mono model for all users or one model per user.

