\subsection{Dealing with Data Updates}
\label{subsec-Ginc}

We next propose \incd to handle $\Delta G$, following the framework in Section~\ref{subsec-framework}.
Given auxiliary structures $\tilde{M}(P,G)$ and \fb,
we put together procedures \identifyaffball, \incmatch and \comb for computing match results for $P$ in $G \oplus \Delta G$.
%
As procedures \incmatch and \comb handle $\Delta G$ basically the same as $\Delta P$, so we mainly show how \identifyaffball identifies \affballsx \wrt $\Delta G$.


\stitle{Procedure \identifyaffball}.
Given $G$, $\Delta G$, and \fb, \identifyaffball identifies \affballsx according to the lemma as follows.

\begin{lemma}
\label{lemma-incgrdata-affballs}
A ball $\ball{[v,r]}$ with center node $v$ in $G$ is identified as an \affballx \wrt $\Delta G$ and \fb,

\sstab{(1)} for some unit data update $\delta$ of $\Delta G$, where
(a) $\delta$ is an edge insertion/deletion, $(w_1,w_2)^+/(w_1,w_2)^-$ and $v$ in  both $\ball{[w_1,r]}$  and $\ball{[w_2,r]}$, or
(b) $\delta$ is a node insertion/deletion, $(w,(w,w'))^+$/$(w)^-$ and $v$ in $\ball{[w,r]}$; or

\sstab{(2)} when $\ball{[v,r]}$ has type code $(1,\ldots,1)$  in \fb.
\end{lemma}

We say a ball which satisfies condition (1) is a {\em structural affected} ball,
\ie the structure of the ball is changed due to the exertion of some updates in $\Delta G$.
%The correctness of procedure \identifyaffball is guaranteed by the following.

\vspace{-0.5ex}
\begin{prop}
\label{prop-affected-datainc}
Given $P$, $G$ and $\Delta G$, if there is a perfect subgraph for $P$ in  ball $\widehat{G\oplus\Delta G}{[v,r]}$ of $G\oplus \Delta G$,
then $\ball{[v,r]}$ must be an affected ball produced by procedure \identifyaffball.
%Given data graph $G$, pattern graph $P$ and data updates $\Delta G$,
%the top-k perfect graphs in $G\oplus \Delta G$ for $P$ must be contained in \affballsx.
\end{prop}

%The proof is deferred to the full version~\cite{fullvldb18}.

Different from pattern updates, procedure \incmatch recomputes partial match relations in $\tilde{M}(P,G)$
for each pattern fragment of $P$ in each {\em structural affected} ball;
and no computation is needed for \affballsx that only satisfy condition (2) in Lemma~\ref{lemma-incgrdata-affballs}.
Procedure \comb combines the partial relations \wrt $\Delta G$ in the same way as handling $\Delta P$.


\stitle{Updating \fb}. Algorithm \incd also updates \fb for all \affballsx.
In addition to updating the links from \fs to \bs in \fb as for pattern updates,
\incd maintains \bs by
(a) removing (resp. inserting new) entries from (resp. to) \bs corresponding to balls whose center nodes are removed from (resp. inserted to) $G$, due to node deletions (resp. node insertions);
and (b) updating the \kw{den} item in \bs \wrt $\Delta G$.
These updates can be done in $O(|\affballsx|)$ time.


\begin{example}
\label{exa-Ginc}
Consider $P_1$ and $G_1$ (both without dashed edges) in Fig.~\ref{fig-motivation-example}, and \fb in Fig.~\ref{fig-auxiliary-structures}(a).
When $\Delta G_1=(\kw{SD_3},\kw{ST_3})^+$ comes,
by Lemma~\ref{lemma-incgrdata-affballs},
\identifyaffball identifies $\ball{[\kw{SD_3}, 2]}$, $\ball{[\kw{ST_3},2]}$, $\ball{[\kw{SA_3}, 2]}$ and $\ball{[\kw{PM_2}, 2]}$ as {\em structural affected} balls,
together with balls with $tc$ $(1, 1)$ in \fb as \affballsx, \ie $\ball{[\kw{PM_1}, 2]}$, while filtering out all other balls.
\end{example}


\stitle{Algorithm \incd.} Given $\tilde{M}(P,G)$, \fb and data updates $\Delta G$,
\incd computes the match results for $P$ in $G \oplus \Delta G$, and maintains
auxiliary structures by invoking procedures \identifyaffball, \incmatch and \comb sequentially.


\stitle{Correctness \& complexity analyses}. The correctness of \incd \wrt $\Delta G$ follows from Proposition~\ref{prop-affected-datainc} and the correctness of \incmatch and \comb.
\incd is overall in
$O(\bigcup_{\widehat{G\oplus\Delta G}\in \affballsx}\bigcup_{i\in[1,h]}$ $r|M(P_{fi}, \widehat{G\oplus\Delta G})|$+$r|P||\affballsx|)$ time \wrt $\Delta G$, while $r$ is small, \ie 2 or 3  (See~\cite{fullvldb18}).
\looseness=-1



\subsection{Unifying Pattern and Data Updates}
\label{subsec-completePG}

We are now ready to provide algorithm \inc, integrating \incp and \incd, presented in Sections~\ref{subsec-Qinc} and~\ref{subsec-Ginc}, respectively, to process continuous pattern and data updates, separately and simultaneously.

Algorithm \inc is able to handle {\em simultaneous} $\Delta P$ and $\Delta G$, because of the consistency in:
(1) the processes for handling $\Delta P$ and $\Delta G$, which follow the same steps in Section~\ref{sec-dynamictopk};
(2) auxiliary data structures for supporting $\Delta P$ and $\Delta G$; and
(3) the combination procedures, which suffice to support simultaneous pattern and data updates.

Observe that \inc can handle {\em continuous} simultaneous $\Delta P$ and $\Delta G$,
as \inc incrementally maintains the auxiliary structures for continuous coming $\Delta P$ and $\Delta G$.




%\vspace{0.5ex}
\stitle{Remark}. Note that (a) the running time of \incp is determined by $P$, $\Delta P$, $\tilde{M}(P,G)$ and \affballsx, not directly depending on $G$, and
(b) the running time of \incd is determined by $P$, $\tilde{M}(P,G)$ and \affballsx, not directly depending on $G$.
From these, we complete the proof of Theorem~\ref{thm-framework-inc}.